\conclusions{

Se presenta un algoritmo completo para el movimiento autónomo basado en campos potecial 
artificial y control de retroalimentación de circuito cerrado. Esto se aplicó a las tareas 
de navegación de un robot de accionamiento diferencial. El controlador de movimiento se 
basa en coordenadas polares. Los resultados del controlador se movimiento demostraron se 
una solución a las restricciones no holonómicas que presenta el robot. De la misma manera, el 
rendimiento del robot en simulación muestra su rápida convergencia a través de trayectorias 
suaves. Las pruebas en un robot Kobuki con transmisión diferencial real muestran resultados 
exitosos. También demuestran la capacidad del robot de movimiento autónomo a pesar de la 
configuración desconocida del entorno. El lidar proporcionó la información del entorno 
para que el robot pueda decidir su movimiento hacia la meta evitando obstáculos en tiempo 
real.

%\begin{enumerate}
%\item Entre las oportunidades más resaltantes se señala el reconocimiento de la
%  importancia de las habilidades informacionales por diversas instituciones
%  educativas, el desarrollo de modelos, normas y directrices en cuanto a
%  alfabetización informacional, y la disponibilidad de aplicaciones web de
%  acceso libre para la creación de cursos virtuales; entre las amenazas se
%  detectan la falta de conocimiento de herramientas de información en los
%  alumnos de educación secundaria, la disponibilidad de gran cantidad de
%  información de baja calidad en la web y el incremento del uso de buscadores
%  como única fuente de información.

%\item Luego de la encuesta aplicada se observa que los alumnos reconocen la
%  importancia del taller, así como el esfuerzo realizado por los
%  bibliotecólogos, sin embargo; perciben que aspectos como la profundidad de
%  los temas, la motivación, el desarrollo de habilidades informativas y el
%  tiempo asignado al taller, pueden ser mejorados.
%\end{enumerate}


}
