\conclusions{

Se presenta un algoritmo para el movimiento autónomo de un robot móvil basado en 
campos potenciales y un control de retroalimentación. Esto se aplicó para la navegación 
de un robot móvil de accionamiento diferencial. El controlador de movimiento se basa en 
coordenadas polares. Los resultados obtenidos para el controlador de movimiento demuestran
que es una buena solución para las restricciones no holonómicas que presenta el robot. De la
misma forma, el rendimiento del robot en simulación muestra su rápida convergencia a través 
de trayectorias suaves. Las pruebas en un robot Kobuki con accionamiento diferencial muestra
resultados bastante exitosos. También demuestran la capacidad del robot de movimiento autónomo
a pesar del entorno que es desconocido. El lidar proporciona la información del entorno
para que el robot pueda decidir su movimiento hacia la meta evitando obstáculos en tiempo real.

Se realizó un sistema mecánico para la construcción de un mapa en tres dimensiones basado en
el sensor lidar, un servomotor y la odometría del robot móvil. Esto se aplicó para el mapeo
tridimensional en una caja, dentro de un túnel prototipo y el pasadizo dentro de un edificio. Se
demuestra que la mayor concentración de mediciones se encuentran en las posiciones reales de 
las paredes dentro del ambiente. El sistema mecánico puede generar un mapa en tres dimensiones 
y dos dimensiones al mismo tiempo. El mapa en dos dimensiones es generado por el algoritmo 
SLAM. Con este mapa se demuestra que el robot móvil tiene la información suficiente para
explorar dentro de un ambiente desconocido.
}
