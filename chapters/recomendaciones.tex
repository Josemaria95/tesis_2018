\recommendations{
	
	Los nuevos métodos de generación de trayectoria pueden ser utilizados para 
	obtener un rango amplio de aplicaciones como rescate de personas, inspección
	de entornos, delivery, etc. Además, las pruebas pueden ser reemplazadas con 
	ambientes de mayor dificultad de acceso para la exploración del robot 
	móvil. Se puede emplear un sensor lidar con mayor cantidad de mediciones por 
	vuelta, ya que esto va a permitir tener una mejor resolución y precisión en 
	la nube de puntos obtenida, beneficiando la exploración autónoma y 
	construcción del mapa tridimensional.

	El robot móvil diferencial, utilizado para las pruebas experimentales, no tiene 
	las características físicas adecuadas para la exploración de socavones 
	subterráneos dentro de las minas subterráneas. El sistema puede ser implementado 
	en otro tipo de robots móviles (terrestre o aéreo). El robot móvil terrestre 
	necesita tener las características adecuadas para movilizarse en superficies 
	bastantes accidentadas y húmedas como son los túneles dentro la mina 
	subterránea. Para un robot móvil aéreo (drone) se necesita que el robot pueda 
	desplazarse a una altura constante y tenga un diseño mecánico adecuado para 
	la humedad del entorno.
}
