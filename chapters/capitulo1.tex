\chapter{INTRODUCCIÓN}

\section{Problemática}

Actualmente, diversos métodos de explotación son utilizados en la minería 
subterránea para la extracción de minerales. Los más utilizados en el Perú 
son variaciones del método de corte y relleno ascendente como descendente. La 
presente tesis se enfoca en el método de corte y relleno ascendente, ya que es el 
más utilizado en el Perú \cite{Cruz2012}. 

En el método de corte y relleno ascendente el mineral es arrancado por franjas 
horizontales y/o verticales empezando por la parte inferior de un tajo y avanzando 
verticalmente. Cuando se ha extraído la franja completa, se rellena el volumen 
correspondiente con material estéril (relleno), que sirve de piso de trabajo a 
los obreros y al mismo tiempo permite sostener las paredes y el 
techo \cite{MunozDelPino2012}. Asimismo, se utiliza grandes máquinas que construyen 
socavones subterráneos por donde se va a extraer los minerales. 

La construcción de socavones subterráneos es de gran importancia para la minería 
subterránea, debido a que por medio de esta construcción se extrae los minerales 
de la mena (roca mineralizada). Estas construcciones son de diversos tamaños  en 
las diferentes unidades mineras del Perú, que dependen del tamaño de las máquinas
que van a ingresar para la extracción del mineral. Durante el proceso de construcción
pueden ocurrir diferentes riesgos (desprendimiento de rocas, emisión de gases, etc.) 
que ponen en peligro la vida de los trabajadores. Para evitar los riesgos dentro de 
la minería subterránea, se necesita supervisar la zona de construcción y a la par 
obtener una representación geométrica visual de toda la excavación. En la minería 
subterránea, para la supervisión y la generación del mapa topográfico los 
especialistas encargados son el topógrafo y el geomecánico. Se dará énfasis a 
la función del topógrafo. 

La función del topógrafo es observar la estructura de los socavones, luego medir 
las dimensiones del área y con los datos obtenidos generar el mapa topográfico. Los 
socavones subterráneos son ambientes demasiados oscuros, esto hace que existan 
limitaciones en la visión humana, por ende los topógrafos realizan un mapa del 
ambiente bastante impreciso. Estos especialistas actualmente utilizan teodolitos 
para medir el área del socavón, pero estas herramientas de medición son 
controladas manualmente, por ende las mediciones no son exactas. El tiempo 
que le toma a cada topógrafo en realizar todo su trabajo es aproximadamente entre 
3 a 4 horas, y posterior a esto el tiempo que se demora para la generación del mapa 
topográfico es entre dos a tres días. Debido a la ineficiencia en el proceso de la 
generación del mapa topográfico, hace que se retrase la producción de extracción 
de minerales dentro de la mina subterránea, ya que los ingenieros mineros 
necesitan calcular el volumen de extracción de la beta mineralizada.



\section{Justificación}


%La propuesta de este trabajo de tesis se centra en desarrollar un robot aéreo autónomo \cite{Lopez2016} para inspeccionar los socavones subterráneos en la mina. Donde se elige un sensor láser para añadirlo al robot aéreo y así poder realizar el vuelo autónomo y el mapa tridimensional del socavón subterráneo.

%Para la inspección de socavones se elige un robot aéreo debido a que brinda muchas ventajas como:
%\begin{itemize}
%\item[•]\textit{Reducción de costos}: Un robot aéreo no necesita de carriles especiales o caminos previamente adaptados para poder ingresar al área que se desea generar el mapa topográfico.
%\item[•]\textit{Reducción de tiempos}: No se necesita ir a una pista de aterrizaje para poner en funcionamiento al robot aéreo, ya que puede funcionar en cualquier parte.
%\item[•]\textit{Zonas de difícil acceso}: Debido a sus dimensiones, maniobrabilidad y a su naturaleza, un robot aéreo puede adentrarse con facilidad en zonas accidentadas, como por ejemplo: grutas, túneles o chimeneas.
%\item[•]\textit{Adaptabilidad de aplicaciones}: Un robot aéreo tiene la posibilidad de adaptarse a diferentes aplicaciones como servicios de vigilancia, mapeos de terreno a tajo abierto, control de incendios, control de infraestructuras eólicas, etc. 
%\item[•]\textit{Reducción de riesgos laborales}: Un robot aéreo autónomo no necesita ser tripulado, por lo tanto puede adentrarse en zonas peligrosas o potencialmente peligrosas evitando el acceso riesgoso de personas. 
%\end{itemize}

%Este robot aéreo utilizará técnicas de SLAM (\textit{Simultaneous Localization And Mapping}) para realizar un mapa del interior del socavón y generar su propia trayectoria dentro del socavón, evitando colisionar con las paredes o los obstáculos del camino de manera autónoma. Se tomará la información del sensor LIDAR (\textit{Light Detection and Ranging}) para generar el mapa topográfico.

%Se elige el sensor LIDAR, ya que emite impulsos rápidos de luz láser, hasta 150.000 impulsos por segundo, sobre una superficie. Este instrumento mide la cantidad de tiempo necesario para que cada pulso pueda recuperarse, y a partir de esto calcula la distancia entre el objeto y el instrumento con una alta precisión. Como utiliza luz láser, el sensor no necesita de iluminación externa para realizar las mediciones, lo cual es perfecto para los socavones de minería subterránea ya que en estas zonas no hay una buena iluminación.

Debido al problema descrito anteriormente, la propuesta de este trabajo de tesis 
se centra en desarrollar un algoritmo de autonomía para la navegaci\'on de un robot 
m\'ovil dentro del socav\'on subterr\'aneo. Donde se elige un sensor l\'aser el 
cual es añadido al robot para su desplazamiento de manera aut\'onoma y a su vez 
construya el mapa tridimensional del socavón subterr\'aneo.

El robot m\'ovil utilizar\'a t\'ecnicas de SLAM (\textit{Simultaneous Localization 
and Mapping}) para realizar un mapa del interior del socav\'on y a su vez poder 
estimar la posici\'on de este dentro de dicho ambiente. Con respecto al sistema 
de navegaci\'on se implementar\'a un algoritmo basado en campos potenciales el 
cual va a permitir que el robot pueda generar su propia trayectoria, evitando 
colisionar con las paredes o los obst\'aculos del camino por donde se va a 
desplazar. Se tomar\'a la informaci\'on del sensor LIDAR (\textit{Light 
Detection and Ranging}) para generar el mapa topogr\'afico en dos dimensiones.

Se elige el sensor LIDAR, ya que emite impulsos r\'apidos de luz l\'aser, hasta 
4000 impulsos por segundo, sobre una superficie. Este instrumento mide la 
cantidad de tiempo necesario para que cada pulso pueda recuperarse, y a partir 
de esto calcula la distancia entre el objeto y el instrumento con una alta 
precisi\'on. Como utiliza luz l\'aser, el sensor no necesita de iluminaci\'on 
externa para realizar las mediciones, lo cual es perfecto para los socavones 
de miner\'ia subterránea ya que en estas zonas no hay una buena iluminaci\'on. 

Para realizar el mapa en tres dimensiones, se diseñar\'a un sistema mec\'anico 
que permitir\'a que el sensor l\'aser pueda tomar mediciones en diferentes 
alturas, obteniendo los datos en las tres coordenadas del sistema de 
referencia. Posteriormente, estos datos serán procesados utilizando un 
algoritmo de ICP (\textit{Iterative Closest Point}) para as\'i obtener una 
nube de puntos. Finalmente, se utilizar\'a un software para poder visualizar 
el mapa tridimensional del socavón subterr\'aneo.

%\section{Objetivos}

%\subsection{General}
%Desarrollar un algoritmo para la navegación autónoma de un robot móvil para inspeccionar los socavones subterráneos en la mina basado en un sistema de sensado láser. 
%\\
%\subsection{Específicos}
%\begin{itemize}
%\item[•]Diseñar e implementar el algoritmo de campos potenciales para un robot móvil.
%\item[•]Diseñar e implementar un controlador para la posición y orientación del robot móvil.
%\item[•]Implementar el método SLAM con el sensor LIDAR.
%\item[•]Configurar la interface del usuario en el sistema operativo ROS (\textit{Robot Operating System}) para la visualización del entorno donde se desplaza el robot móvil.
%\end{itemize}

\section{Alcances y limitaciones}

Para realizar el mapa topogr\'afico se debe considerar que el t\'unel subterr\'aneo 
tiene un terreno bastante accidentado y el robot m\'ovil donde fue implementado el 
algoritmo no tiene el diseño mec\'anico correspondiente para este tipo de ambiente. Por 
tal motivo el presente trabajo ser\'a capaz de moverse por un t\'unel con una superficie 
bastante recta sin cambios bruscos en la pendiente. Otra dificultad a considerar es la 
dimensi\'on de los socavones que var\'ian seg\'un el tama\~no de la beta mineralizada, por 
ende el trabajo de tesis ser\'a capaz de mapear el t\'unel que no exceda los 16 metros de 
ancho debido al rango de alcance que tiene el sensor lidar. 

Finalmente, el algoritmo de autonom\'ia puede usarse para cualquier robot m\'ovil (terrestre 
o a\'ereo) considerando que el modelo din\'amico y cinem\'atico de estos ya se encuentran 
previamente implementados. Para el caso del robot m\'ovil a\'ereo se debe tener en cuenta 
que no debe haber variaciones bruscas en la altura, se debe mantener un vuelo estable.


%Para la realizaci\'on de un mapa topogr\'afico dentro del t\'unel subterr\'aneo se presenta dificultades como: las dimensiones de esta estructura y el terreno bastante accidentado por donde se tiene que desplazar para realizar las mediciones. Por tales motivos el alcance y las limitaciones son:

%\subsection{Alcances}
%\begin{itemize}
%	\item[•] El presente trabajo ser\'a capaz de mapear un t\'unel donde el suelo se mantenga recto sin pendientes y que tenga dos divisiones.

%	\item[•] El algoritmo de autonom\'ia se podr\'a implementar en cualquier robot móvil considerando que la dinámica y la cinemática ya se encuentran implementadas.
%\end{itemize} 

%\subsection{Limitaciones}
%\begin{itemize}
%	\item[•] El presente trabajo ser\'a capaz de mapear un t\'unel que no exceda los 16 metros de ancho.

%	\item[•] El robot m\'ovil donde fue implementado el algoritmo no puede ser utilizado en un socav\'on subterr\'aneo debido a su estructura mec\'anica.
%\end{itemize} 
\section{Organizaci\'on de la tesis}

