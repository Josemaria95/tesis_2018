\chapter{INTRODUCCIÓN}

\section{Problemática}

Actualmente, diversos métodos de explotación son utilizados en la minería 
subterránea para la extracción de minerales. Los más utilizados en el Perú 
son variaciones del método de corte y relleno ascendente como descendente. La 
presente tesis se enfoca en el método de corte y relleno ascendente, ya que es el 
más utilizado en el Perú \cite{Cruz2012}. 

En el método de corte y relleno ascendente el mineral es arrancado por franjas 
horizontales y/o verticales empezando por la parte inferior de un tajo y avanzando 
verticalmente. Cuando se ha extraído la franja completa, se rellena el volumen 
correspondiente con material estéril (relleno), que sirve de piso de trabajo a 
los obreros y al mismo tiempo permite sostener las paredes y el 
techo \cite{MunozDelPino2012}. Asimismo, se utiliza grandes máquinas que construyen 
socavones subterráneos por donde se va a extraer los minerales. 

La construcción de socavones subterráneos es de gran importancia para la minería 
subterránea, debido a que por medio de esta construcción se extrae los minerales 
de la mena (roca mineralizada). Estas construcciones son de diversos tamaños  en 
las diferentes unidades mineras del Perú, que dependen del tamaño de las máquinas
que van a ingresar para la extracción del mineral. Durante el proceso de construcción
pueden ocurrir diferentes riesgos (desprendimiento de rocas, emisión de gases, etc.) 
que ponen en peligro la vida de los trabajadores. Para evitar los riesgos dentro de 
la minería subterránea, se necesita supervisar la zona de construcción y a la par 
obtener una representación geométrica visual de toda la excavación. En la minería 
subterránea, para la supervisión y la generación del mapa topográfico los 
especialistas encargados son el topógrafo y el geomecánico. Se dará énfasis a 
la función del topógrafo. 

La función del topógrafo es observar la estructura de los socavones, luego medir 
las dimensiones del área y con los datos obtenidos generar el mapa topográfico. Los 
socavones subterráneos son ambientes demasiados oscuros, esto hace que existan 
limitaciones en la visión humana, por ende los topógrafos realizan un mapa del 
ambiente bastante impreciso. Estos especialistas actualmente utilizan teodolitos 
para medir el área del socavón, pero estas herramientas de medición son 
controladas manualmente, por ende las mediciones no son exactas. El tiempo 
que le toma a cada topógrafo en realizar todo su trabajo es aproximadamente entre 
3 a 4 horas, y posterior a esto el tiempo que se demora para la generación del mapa 
topográfico es entre dos a tres días. Debido a la ineficiencia en el proceso de la 
generación del mapa topográfico, hace que se retrase la producción de extracción 
de minerales dentro de la mina subterránea, ya que los ingenieros mineros 
necesitan calcular el volumen de extracción de la beta mineralizada.



\section{Justificación}

Debido al problema descrito anteriormente, la propuesta de este trabajo de tesis 
se centra en desarrollar un algoritmo de autonomía para la navegaci\'on de un robot 
m\'ovil dentro del socav\'on subterr\'aneo. Donde se elige un sensor l\'aser el 
cual es añadido al robot para su desplazamiento de manera aut\'onoma y a su vez 
construya el mapa tridimensional del socavón subterr\'aneo.

El robot m\'ovil utilizar\'a t\'ecnicas de SLAM (\textit{Simultaneous Localization 
and Mapping}) para realizar un mapa del interior del socav\'on y a su vez poder 
estimar la posici\'on de este dentro de dicho ambiente. Con respecto al sistema 
de navegaci\'on se implementar\'a un algoritmo basado en campos potenciales el 
cual va a permitir que el robot pueda generar su propia trayectoria, evitando 
colisionar con las paredes o los obst\'aculos del camino por donde se va a 
desplazar. Se tomar\'a la informaci\'on del sensor LIDAR (\textit{Light 
Detection and Ranging}) para generar el mapa topogr\'afico en dos dimensiones.

Se elige el sensor LIDAR, ya que emite impulsos r\'apidos de luz l\'aser, hasta 
4000 impulsos por segundo, sobre una superficie. Este instrumento mide la 
cantidad de tiempo necesario para que cada pulso pueda recuperarse, y a partir 
de esto calcula la distancia entre el objeto y el instrumento con una alta 
precisi\'on. Como utiliza luz l\'aser, el sensor no necesita de iluminaci\'on 
externa para realizar las mediciones, lo cual es perfecto para los socavones 
de miner\'ia subterránea ya que en estas zonas no hay una buena iluminaci\'on. 

Para realizar el mapa en tres dimensiones, se diseñar\'a un sistema mec\'anico 
que permitir\'a que el sensor l\'aser pueda tomar mediciones en diferentes 
alturas, obteniendo los datos en las tres coordenadas del sistema de 
referencia. Posteriormente, estos datos serán procesados utilizando un 
algoritmo de ICP (\textit{Iterative Closest Point}) para as\'i obtener una 
nube de puntos. Finalmente, se utilizar\'a un software para poder visualizar 
el mapa tridimensional del socavón subterr\'aneo.

\section{Alcances y limitaciones}

Los mapas topográficos son realizados dentro de un túnel subterráneo, este ambiente 
tiene un terreno bastante accidentado lleno de rocas y charcos de agua. Se debe 
considerar que el robot móvil donde el algoritmo fue implementado no tiene el 
diseño mecánico adecuado para este entorno. Por ende la tesis será capaz de 
moverse a través de una superficie plana sin cambios bruscos.

Las dimensiones del túnel subterráneo es otra dificultad a considerar. Los túneles
varían según el tamaño de la beta mineralizada y según el proceso de explotación. Por 
tal motivo, la tesis será capaz de realizar mediciones dentro de un túnel 
que no exceda los 16 metros de diámetro que tiene como rango el sensor lidar.

Finalmente, el algoritmo de autonom\'ia puede usarse para cualquier robot 
m\'ovil (terrestre o a\'ereo) considerando que el modelo din\'amico y cinem\'atico 
de estos fueron previamente implementados. Para la implementación del algoritmo 
de autonomía en un robot aéreo, se debe considerar que el robot debe volar a una 
altura constante y a su vez mantener un vuelo estable.

\section{Organizaci\'on de la tesis}

El documento está organizado en 4 capítulos los cuales serán explicados en esta
sección. En el capítulo 1 se explicará la problemática a resolver, la justificación y 
el alcance del presente trabajo de tesis. En el capítulo 2 se explicará los algoritmos 
que se usan actualmente para el control de movimiento del robot móvil, la generación 
de trayectoria y  los algoritmos SLAM que existen. En el capítulo 3 se describirá la 
metodología que se hizó para el control de movimiento, la generación de trayectoria y 
la construcción del mapa tridimensional. En el capítulo 4 se describirá y explicará 
con detalle los resultados obtenidos.Se realizó pruebas en simulación y en un ambiente 
real asemejándose a la estructura de un túnel subterráneo. Finalmente, se describé las 
conclusiones de la presente tesis.