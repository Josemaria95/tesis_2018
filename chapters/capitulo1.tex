\chapter{Introducción}

En este primer capítulo se presenta una introducción a este trabajo de 
tesis. En la primera sección se comienza explicando la problemática existente 
en las empresas mineras, que trabajan con túneles subterráneos, y se termina 
explicando la motivación para desarrollar este trabajo. En las siguientes secciones se explican 
la propuesta de solución seguido de los alcances y limitaciones de esta tesis. Finalmente, se termina 
el capítulo con una descripción de la organización general de este documento.

\section{Problemática}

Actualmente, en la minería subterránea se utilizan diversos métodos de explotación para la 
extracción de minerales. Los más utilizados, en el Perú, son variaciones del método de 
corte y relleno ascendente como descendente. La presente tesis se enfoca en el método 
de corte y relleno ascendente, ya que es el más utilizado en el Perú \cite{Cruz2012}. En 
el método de corte y relleno ascendente, el mineral es arrancado por franjas horizontales 
y verticales empezando por la parte inferior de un tajo y avanzando verticalmente. Cuando 
se ha extraído la franja completa, se rellena el volumen correspondiente con material 
estéril, llamado ``relleno", que sirve de piso de trabajo a los obreros y al mismo tiempo permite 
sostener las paredes y el techo \cite{MunozDelPino2012}. Asimismo, se utiliza grandes 
máquinas que construyen túneles subterráneos por donde se va a extraer los minerales. 

La construcción de túneles subterráneos es de gran importancia para la minería subterránea, 
debido a que por medio de esta construcción se extrae los minerales de la veta mineralizada 
\cite{EtapaTunelSubterraneo}. Estas construcciones son de diversos tamaños en las diferentes 
unidades mineras del Perú, y dependen del tamaño de las máquinas que van a ingresar para la 
extracción del mineral. Durante el proceso de construcción pueden ocurrir diferentes accidentes 
(desprendimiento de rocas, emisión de gases, etc.) que ponen en peligro la vida de los 
trabajadores \cite{GasesMinero}. Para evitar los riesgos dentro de la minería subterránea, se 
necesita supervisar la zona de construcción y a su vez obtener una representación geométrica 
visual de toda la excavación \cite{SeguridadMinera}. Los especialistas encargados de la supervisión 
y la generación visual de un mapa topográfico del ambiente, dentro de la minería subterránea, son 
el topógrafo y el geomecánico. Este trabajo de tesis se inspira en las principales funciones del 
topógrafo dentro de la mina. 

%La motivación de esta tesis se centra en hacer más eficiente la labor del topógrafo. 
El topógrafo observa y mide las dimensiones del túnel subterráneo, para generar un mapa 
topográfico de éste. El túnel subterráneo es un ambiente bastante oscuro, el 
cual limita la visión del especialista generando imprecisión en sus mediciones y por ende en 
los mapas que generan. Esto es un problema en los mapas elaborados de estos ambientes. Para 
realizar la toma de mediciones dentro del túnel, el topógrafo se demora entre tres a cuatro 
horas y para generar un mapa en tres dimensiones, del lugar, se demoran entre dos a tres 
días. La demora en las funciones del topógrafo genera un retraso en el proceso de la extracción 
de minerales y a su vez proporciona poca información sobre la extracción de la veta 
mineralizada, por día. Las mineras ganan dinero por la cantidad de volumen extraído y asimismo
paga a sus trabajadores por volumen trabajado. Por ende, la motivación de esta tesis es hacer 
más eficiente la labor del topógrafo apoyándose en el uso de la tecnología. 

%Las funciones principales del topógrafo son observar, medir y generar el mapa topográfico 
%del túnel subterráneo. 
%La motivación de esta tesis se centra en la labor del topógrafo. Este especialista
%se encarga de observar, medir y generar un mapa topográfico de los túneles 
%subterráneos. Debido a que los túneles subterráneos son bastante oscuros, el topógrafo
%presenta limitaciones en su visión generando imprecisión en sus mapas. 

%La función del topógrafo es observar la estructura de los socavones, luego medir 
%las dimensiones del área y con los datos obtenidos generar el mapa topográfico. Los 
%socavones subterráneos son ambientes demasiados oscuros, esto hace que existan 
%limitaciones en la visión humana, por ende los topógrafos realizan un mapa del 
%ambiente bastante impreciso. Estos especialistas actualmente utilizan teodolitos 
%para medir el área del socavón, pero estas herramientas de medición son 
%controladas manualmente, por ende las mediciones no son exactas. El tiempo 
%que le toma a cada topógrafo en realizar todo su trabajo es aproximadamente entre 
%3 a 4 horas, y posterior a esto el tiempo que se demora para la generación del mapa 
%topográfico es entre dos a tres días. Debido a la ineficiencia en el proceso de la 
%generación del mapa topográfico, hace que se retrase la producción de extracción 
%de minerales dentro de la mina subterránea, ya que los ingenieros mineros 
%necesitan calcular el volumen de extracción de la beta mineralizada.



\section{Justificación}

Debido al problema descrito en la sección anterior, este trabajo de tesis se centra 
en el desarrollo e implementación de un sistema de localización y mapeo en tres 
dimensiones basado en sensado láser para la exploración autónoma de un robot móvil.

Este sistema busca ofrecer una alternativa tecnológica de medición y la construcción 
tridimensional de un túnel subterráneo mediante el uso de un sensor láser. Además, la
construcción se realiza mientras un robot móvil va explorando el ambiente. El 
desarrollo de este sistema beneficia a las personas que se encuentran involucradas 
dentro del proceso de extracción del mineral, debido a que el sistema reduce el tiempo 
que se demora un topógrafo en realizar el mapa tridimensional. Además, el sistema permite
tener un mejor seguimiento a la producción de mineral al permitir conocer los volúmenes
de mineral que se extrae cada período de tiempo durante un día. Asimismo, este sistema 
evita que el topógrafo se exponga a riesgos dentro de la minería subterránea. Al evitar 
estos tipos de accidentes se reduce costos y en caso de ocurrir algún imprevisto
grave, el daño sería de forma directa hacia el robot y no hacia una persona.
%El robot móvil utiliza técnicas de SLAM (\textit{Simultaneous Localization and Mapping}) 
%para realizar un mapa en dos dimensiones dentro del túnel subterráneo y asimismo, estimar
%la posición del robot móvil dentro del ambiente mencionado. Se implementó un sistema de 
%navegación basado en campos potenciales. Este algoritmo permite al robot generar su 
%propia trayectoria, evitando colisionar con las paredes u obstáculos en el ambiente 
%donde comenzará a explorar. Se toma la información del sensor lidar (\textit{Light
%Detection and Ranging}) para generar el mapa en dos dimensiones.
%Debido al problema descrito anteriormente, la propuesta de este trabajo de tesis 
%se centra en desarrollar un algoritmo de autonomía para la navegaci\'on de un robot 
%m\'ovil dentro del socav\'on subterr\'aneo. Donde se elige un sensor l\'aser el 
%cual es añadido al robot para su desplazamiento de manera aut\'onoma y a su vez 
%construya el mapa tridimensional del socavón subterr\'aneo.

%El robot m\'ovil utilizar\'a t\'ecnicas de SLAM (\textit{Simultaneous Localization 
%and Mapping}) para realizar un mapa del interior del socav\'on y a su vez poder 
%estimar la posici\'on de este dentro de dicho ambiente. Con respecto al sistema 
%de navegaci\'on se implementar\'a un algoritmo basado en campos potenciales el 
%cual va a permitir que el robot pueda generar su propia trayectoria, evitando 
%colisionar con las paredes o los obst\'aculos del camino por donde se va a 
%desplazar. Se tomar\'a la informaci\'on del sensor LIDAR (\textit{Light 
%Detection and Ranging}) para generar el mapa topogr\'afico en dos dimensiones.

%Se elige el sensor lidar ya que emite impulsos r\'apidos de luz l\'aser hasta 
%4000 impulsos por segundo, sobre una superficie. Este instrumento mide la 
%cantidad de tiempo necesario para que cada pulso pueda recuperarse, y a partir 
%de esto calcula la distancia entre el objeto y el instrumento con una alta 
%precisi\'on. Como utiliza luz l\'aser, el sensor no necesita de iluminaci\'on 
%externa para realizar las mediciones, lo cual es perfecto para ambientes
%donde no hay una buena iluminaci\'on. 

%Finalmente, para construir el mapa en tres dimensiones se diseñó un sistema 
%mecánico que permite rotar el sensor láser haciendo que este pueda tomar 
%medidas a diferentes alturas con su respectivo ángulo de inclinación. Los datos
%obtenidos son enviados a una matriz heterógenea, el cual nos genera los valores
%en los ejes $(\X,\y,\z)$. El mapa tridimensional obtenido se puede visualizar en un 
%software de fácil interacción.

%Para realizar el mapa en tres dimensiones, se diseñar\'a un sistema mec\'anico 
%que permitir\'a que el sensor l\'aser pueda tomar mediciones en diferentes 
%alturas, obteniendo los datos en las tres coordenadas del sistema de 
%referencia. Posteriormente, estos datos serán procesados utilizando un 
%algoritmo de ICP (\textit{Iterative Closest Point}) para as\'i obtener una 
%nube de puntos. Finalmente, se utilizar\'a un software para poder visualizar 
%el mapa tridimensional del socavón subterr\'aneo.

\section{Alcances y limitaciones}

La motivación, como se mencionó antes, es hacer más eficiente la labor del topógrafo 
dentro de un túnel subterráneo. Este proyecto de tesis es un prueba de concepto para 
demostrar que el sistema desarrollado funciona para explorar un ambiente desconocido y 
a su vez generar un mapa en tres dimensiones del lugar.

Los túneles subterráneos son ambientes con el terreno bastante accidentado, lleno de
rocas y charcos de agua. Para este trabajo de tesis no se considera el desarrollo de 
un robot móvil para que se pueda desplazar en terrenos accidentados, como los túneles 
subterráneos de la minería. El objetivo de la tesis es el desarrollo de un sistema 
de exploración para un robot móvil. 
%Este sistema puede ser implementado en un robot móvil (terrestre o aéreo), previamente 
%implementado el módelo dinámico y cinemático. Para implementar el sistema de exploración 
%en un robot móvil aéreo, se debe considerar una altura constante de vuelo y un vuelo 
%estable, sin cambios bruscos.

Finalmente, el sistema de exploración se implementó en un robot móvil diferencial, el 
cual debe de desplazarse dentro de una superficie plana. Se utilizó un sensor lídar, con
un rango máximo de medición de 6 metros. El prototipo en conjunto es capaz de realizar 
mediciones dentro de un ambiente que no exceda los 12 metros de diámetro.

%Los mapas topográficos son realizados dentro de un túnel subterráneo, este ambiente 
%tiene un terreno bastante accidentado lleno de rocas y charcos de agua. Se debe 
%considerar que el robot móvil donde el algoritmo fue implementado no tiene el 
%diseño mecánico adecuado para este entorno. Por ende la tesis será capaz de 
%moverse a través de una superficie plana sin cambios bruscos.

%Las dimensiones del túnel subterráneo es otra dificultad a considerar. Los túneles
%varían según el tamaño de la beta mineralizada y según el proceso de explotación. Por 
%tal motivo, la tesis será capaz de realizar mediciones dentro de un túnel 
%que no exceda los 16 metros de diámetro que tiene como rango el sensor lidar.

%Finalmente, el algoritmo de autonom\'ia puede usarse para cualquier robot 
%m\'ovil (terrestre o a\'ereo) considerando que el modelo din\'amico y cinem\'atico 
%de estos fueron previamente implementados. Para la implementación del algoritmo 
%de autonomía en un robot aéreo, se debe considerar que el robot debe volar a una 
%altura constante y a su vez mantener un vuelo estable.

\section{Organizaci\'on de la tesis}

El documento está organizado en 4 capítulos, los cuales serán explicados en esta
sección. En el capítulo 1 se presenta la problemática a resolver, la justificación y 
el alcance del presente trabajo de tesis. En el capítulo 2 se explica los algoritmos 
que se usan actualmente para el control de movimiento del robot móvil, la generación 
de trayectoria y  los algoritmos SLAM (Simultaneous Localization And Mapping) que 
existen. En el capítulo 3 se describe la metodología que se hizo para el control de 
movimiento, la generación de trayectoria y la construcción del mapa tridimensional. En 
el capítulo 4 se muestra y explica con detalle los resultados obtenidos. Se 
realizaron pruebas en simulación y en un ambiente real asemejándose a la estructura de 
un túnel subterráneo. Finalmente, se describe las conclusiones de la presente tesis.